\documentclass[ignorenonframetext]{beamer}
\usepackage[english,russian]{babel}
\usepackage[utf8]{inputenc}
\usetheme{Dresden} % Выбираем основную тему
\usefonttheme{serif} % Дополнительно подключаем шрифты
% Авторы
\author{Вилинский Д.А.}
% Название презентации
\title{Создание презентациии в LaTeX}
% Организации
\institute{УрФУ}
% Дата (команда \today выводит текущую дату)
\date{\today}

\begin{document}
% Титульная страница
\frame{\titlepage}
% Содержание презентации (выводятся только основные разделы)
\frame{\frametitle{Содержание}\tableofcontents} 
\chapter{Содержание презентации}
\section{основное о Beamer}
\begin{frame}{Beamer}
\begin{itemize}
\item Beamer — класс для LATEX’а, специально предназначенный
для создания слайдов и презентаций
\item Позволяет контролировать внешний вид, цвета, темы,
переход между слайдами и т.п.
\item Добавляет некоторые новые возможности к уже
привычным командам TEX’а
\end{itemize}
\end{frame}

\begin{frame}{Где взять?}
\begin{itemize}
\item Скорее всего Beamer уже есть в поставке LATEX’а
\item Если нет, то можно скачать с
http://bitbucket.org/rivanvx/beamer/wiki/Home
\item Там же можно взять крайне полезное руководство
пользователя Beamer
\end{itemize}
\end{frame}
\section{Начало создания}

\begin{frame}{Начанаем}
Первый шаг — задать базовую информацию о презентации:
название, автора, организацию и т.п.\\
\begin{block}{В преамбуле}
$\backslash$title[short title]\{long title\}\\
$\backslash$subtitle[short subtitle]\\\
$\backslash$author[short name]\{long name\}\\
$\backslash$date[short date]\{long date\}\\
$\backslash$institution[short name]\{long name\}\\
\end{block}
\end{frame}

\begin{frame}
\frametitle{Frame’ы}
\begin{itemize}
\item Каждая презентация состоит из кадров (frame’ов)
\item Каждый frame может произвести один и более слайдов (в
зависимости от переходов-overlay’ев)
\end{itemize}
\\
\begin{block}{Просто кадр}
$\backslash$begin{frame}[<alignment>]\\
$\backslash$frametitle\{Frame Title Goes Here\}\\
Frame body text and/or LATEX code\\
$\backslash$end\{frame\}
\end{block}
\end{frame}

\begin{frame}
\frametitle{Титульный Слайд}
На титульном слайде отображается все то, что было задано в
преамбуле\\


\begin{block}{Титульный Слайд}
$\backslash$begin\{frame\}\\
$\backslash$titlepage\\
$\backslash$end\{frame\}\\
\end{block}
\end{frame}


\section{Структурирование}
\begin{frame}
\frametitle{Структурирование}
Beamer предоставляет множество способов структурирования
информации на слайдах. Мы остановимся на
\begin{itemize}
\item  Многоколоночной верстке
\item  Блоках
\end{itemize}
\end{frame}

\begin{frame}
\frametitle{Многоколоночная верстка}
Для набора текста на слайдах в колонки служит окружение
columns\\
\begin{block}{пример}
$\backslash$begin\{columns\}\\
$\backslash$column\{.3$\backslash$textwidth\}\\
Левый столбец\\
$\backslash$column\{.3$\backslash$textwidth\}\\
Правый столбец\\
$\backslash$end\{columns\}\\

\begin{columns}
\column{.3\textwidth}
Левый столбец
\column{.3\textwidth}
Правый столбец
\end{columns}
\end{block}
\end{frame}

\begin{frame}
\frametitle{Блоки}
Блоки служат для выделения отдельных фрагментов текста:

\begin{block}{Пример}
$\backslash$begin\{block\}\{Пример\}\\
Немного рекурсии\\
$\backslash$end\{block\}\\
\end{block}
\end{frame}

\chapter{Оформление}
\section{Оформление и цветовые темыы}

\begin{frame}
\frametitle{Темы}
\begin{itemize}
\item Темы могут полностью поменять внешний вид
презентации.
\item Тема презентации может быть выбрана посредством
$\backslash$usetheme\{ThemeName\}
\end{itemize}

\begin{block}{Доступные темы}
Antibes Bergen Berkeley Berlin Boadilla Copenhagen Darmstadt
Dresden Frankfurt Goettingen Hannover Ilmenau Juanlespins
Madrid Malmoe Marburg Montpellier Paloalto Pittsburgh
Rochester Singapore ...
\end{block}
\end{frame}

\begin{frame}
\frametitle{Цветовые Темы}
\begin{itemize}
\item  Темы могут полностью поменять внешний вид
презентации.
\item   Тема презентации может быть выбрана посредством
$\backslash$usetheme\{ThemeName\}

\end{itemize}

\begin{block}{Доступные цветовые темы}
albatross crane beetle dove fly seagull wolverine beaver
\end{block}
\end{frame}

\end{document}