\documentclass{article}
\usepackage[T2A]{fontenc}
\usepackage[utf8]{inputenc}
\usepackage[russian]{babel}
\addto\captionsrussian{\def\refname{Список используемой литературы}}

\title{Dнутренние библиографии LaTEX}
\author{Дмитрий Вилинский}
\date{June 2019}

\begin{document}
\maketitle

\section{Tеоретическая часть}
Текст взят из \cite{Lat1}\\
\LaTeX{}  предоставляет возможность оформить список литературы, элементы которого нумеруются автоматически; в тексте при этом надо ссылаться не на эти номера, которые могут измениться в процессе работы над документом, а на установленные Вами условные обозначения для элементов списка литературы (будем для краткости называть их «источниками»).
Список литературы оформляется как окружение thebibliography. Это окружение имеет обязательный аргумент — номер элемента библиографии, который займет больше всего места на печати (в стандартных шрифтах все цифры имеют одинаковую ширину, так что достаточно привести в качестве аргумента, например, номер 99, если в списке литературы будет заведомо меньше 100 источников).
Каждый элемент списка литературы вводится командой $\backslash$bibitem. У нее есть один обязательный аргумент — Ваше условное обозначение. В качестве такого обозначения можно использовать любую последовательность из букв и цифр. В тексте ссылка на элемент списка литературы делается с помощью команды $\backslash$cite. У нее есть обязательный аргумент — условное обозначение того источника, на который Вы ссылаетесь. Можно сослаться сразу на несколько источников — для этого в аргументе ко
манды $\backslash$cite надо указать через запятую обозначения для тех источников, на которые Вы
хотите сослаться. 

Следующий матерал взят с  \cite{Lat2}
переименовываем  список литературы в "список используемой литературы"\\
$\backslash$addto$\backslash$captionsrussian\{$\backslash$def$\backslash$refname\{Список используемой литературы\}\}

\begin{thebibliography}{99}
\bibitem{Lat1} С.М.Львовский "LATEX: подробное описание"
\bibitem{Lat2} http://fkn.ktu10.com/?q=node/6860
\end{thebibliography}
\end{document}